
\documentclass{article}
\usepackage[utf8]{inputenc}
\usepackage[bulgarian]{babel}

\usepackage{systeme}
\usepackage{amsmath}

\usepackage{cmap}
\usepackage[utf8]{inputenc}
\usepackage[T2A]{fontenc}

\newtheorem{definition}{Дефиниция}

\usepackage{comment}
%\use
\newtheorem{problem}{Задача}

\newcounter{solution}

\usepackage{graphicx}
\usepackage{pst-plot}

\usepackage{tikz}

\newcommand\solution{%
	\stepcounter{solution}%
	\textbf{Решение :}\\%
}


\date{}

\title{План анализ(mission possible)}
\begin{document}
	
	\maketitle
	Записваме подолу план за действие за вземане на изпит. A lot'a work needed for the mission to succeed.
	\begin{enumerate}
		
		\item  Тема 1 - не се учи
		\item Тема 2 - не се учи
		\item Тема 3 - не се учи
		\item Тема 4 - не се учи
		\item Редица, сходяща редица, околност.
		\begin{itemize}
		\item сх. редица е ограничена
		\item премахване краен брой членове не променя границата
		\item $a_n \pm  b_n \to a\pm b$
		\item Лема за полицаите(5.11)
		\item Монотонно растяща и ограничена $\to $ сходяща
	\end{itemize}
		
		\item подредица, редица на Коши
		\begin{itemize}
			\item сх. редица $\to$ подредица има същата граница
			\item редица на Коши $\iff$ сходяща
	
		\end{itemize}
	\item Темата е доказтелство на:\\
	$e = 1 + \frac{1}{2!} + \frac{1}{3!} + ... $ и \\
	$e = \lim (1+\frac{1}{n})^n $
	
	\item ред, частична сума
	\begin{itemize}
		\item от редици се прехвърлят няколко твърдения
		\item $\sum \frac{1}{n}, \sum \frac{1}{n^2} $, геометр. прогресия
	\end{itemize}
	\item (не се учи, ако не знаеш предишната), абсолютно сходящ ред условно сходящ ред
	\begin{itemize}
		\item $\sum a_n b_n, |a_n|<\infty, b_n \to 0 $, e сходящ
		\item ред с алтернативно сменящи се знаци е сходящ
		\item какво става ако разместим членовете на условно сходящ/ абсолютно сходящ ред?
	\end{itemize}
	\item Умножаване на безкрайни редове
	\begin{itemize}
		\item $\sum a_n b_n$, при $\sum|a_n|<\infty$, $\sum b_n <\infty$, e сходящ
		\item $e^x $ ред ма Тейлор
		\end{itemize}
	\item елементарни функции, растяща(монотонно), периодична.(Няма да се даде, няма нищо в тази тема)
	\item Граници по Хайне, Коши. еквивалентност, точка на сгъстяване
	\begin{itemize}
		\item много определения за клонене на функция, $О(f)$ и $o(f).$
	\end{itemize}
	
	
	\item $\lim_{x\to0} \frac{sin(x)}{x} = 1  $, $\lim_{x\to0} \frac{e^x-1}{x} = 1  $, темата е само сметки и приложения на определенията(нисък приоритет)
	
	\item (don't know whether to study or not) Непрекъсната фунцкия, сума,разлика,композиция на непрекъснати функции, равномерна непрекъснатост.
	\item Вайерщраски 1,2,3, обратна функция
	\begin{itemize}
		\item Непрекъсната функция в затворен интервал достига най-голямата и най-малката си стойност.
		\item $f(a) < 0, f(b)>0 $ за $f$ дефинирана в $[a,b]$ има корен
		\item непрекъсната функция в интервал е обратима
	\end{itemize}
	\item (don't know whether to study or not)$e^x$ - дефиниция и свойства(7 на брой), същото за логаритъм
	\item (don't know whether to study or not) в кои интервали тригонометричните функции са обратими? обратни тригонометрични функции
	
	\item (!) Производна
	\begin{enumerate}
		\item $\exists f' \to f$ е непр.
		\item Производна на $\pm$,  $.$, $/$.
		\item Производна на полином
		
	\end{enumerate}
	\item производна на сложна, обратна, $ln, arcsin$
	\item (!) Локален min,max
	\begin{itemize}
		\item Ферма
		\item Рол
		\item Лагранж
		\item Коши
	\end{itemize}
	\item n-ta производна, Формула на Тейлор
	\begin{itemize}
		\item $\exists f^{n+1} \to \exists$ ред на Taylor 
		\item $ sin, cos$
	\end{itemize}
	\item изпъкнала, вдлъбната функция
	\begin{enumerate}
		\item $f''>0 \to$ изпъклана
		\item $f$ изпъкнала$\to f(\sum(\lambda_i a_i)) \leq \sum \lambda_i f(a_i) $
	\end{enumerate}
	\item Лопитал(L'Hopital or L' (not) Ho(s)pital for newcommers), (no study)
	\item The master plan is here! It was here all along!
	\item Analytic Number theory for newborn(mathematicians) - (no study)
	\begin{itemize}
		\item алгебричните числа са изброимо м-во
		\item Число на Лювил
		\item Число на Ойлер
	\end{itemize}
	\item (!) Integrals(integrate 
	'em all in your head!)
	\begin{enumerate}
		\item Дефиниция на примитивна, неопр. интеграл
		\item основни интеграли
		\item $\int f' = f$, $\int (\lambda f + \mu g) = \lambda \int f + \mu \int g $
		\item интегриране по части, внасяне под диференциала
	\end{enumerate}
	\item Are you thinking rationally about functions? Or are you thinking about rational functions? (not for study)
	\begin{enumerate}
		\item Theorem
	\end{enumerate}
	\item no study
	\item no study
	
	
	
	\end{enumerate}
	
	
	
	
	
	
	
	
	
\end{document}	
	
	

