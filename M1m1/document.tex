\documentclass{article}
\usepackage[utf8]{inputenc}
\usepackage[bulgarian]{babel}

\usepackage{systeme}
\usepackage{amsmath}
\usepackage{amsfonts}

\usepackage{cmap}
\usepackage[utf8]{inputenc}
\usepackage[T2A]{fontenc}

\newtheorem{definition}{Дефиниция}

\usepackage{comment}
%\use
\newtheorem{problem}{Задача}
\newtheorem{theorem}{Теорема}
\newtheorem{example}{Пример}
\newcounter{solution}

\usepackage{graphicx}
\usepackage{pst-plot}

\usepackage{tikz}

\newcommand\solution{%
	\stepcounter{solution}%
	\textbf{Решение :}\\%
}

\usetikzlibrary{angles,
	quotes}
\usepackage{siunitx}

% preparation for gaussian elimination
%%%%%%%%%%%%%%%%%%
\usepackage[margin=1in]{geometry}
\usepackage{mathtools}
\usepackage{array}

\makeatletter
\newcounter{elimination@steps}
\newcolumntype{R}[1]{>{\raggedleft\arraybackslash$}p{#1}<{$}}
\def\elimination@num@rights{}
\def\elimination@num@variables{}
\def\elimination@col@width{}
\newenvironment{elimination}[4][0]
{
	\setcounter{elimination@steps}{0}
	\def\elimination@num@rights{#1}
	\def\elimination@num@variables{#2}
	\def\elimination@col@width{#3}
	\renewcommand{\arraystretch}{#4}
	\start@align\@ne\st@rredtrue\m@ne
}
{
	\endalign
	\ignorespacesafterend
}
\newcommand{\eliminationstep}[2]
{
	\ifnum\value{elimination@steps}>0\sim\quad\fi
	\left[
	\ifnum\elimination@num@rights>0
	\begin{array}
		{@{}*{\elimination@num@variables}{R{\elimination@col@width}}
			|@{}*{\elimination@num@rights}{R{\elimination@col@width}}}
		\else
		\begin{array}
			{@{}*{\elimination@num@variables}{R{\elimination@col@width}}}
			\fi
			#1
		\end{array}
		\right]
		& 
		\begin{array}{l}
			#2
		\end{array}
		&%                                    moved second & here
		\addtocounter{elimination@steps}{1}
	}
	\makeatother


%%%%%



%\date{}

\title{Хакване на някаква математика в УНСС}
\begin{document}
	\maketitle
	
	\section{Теми и коментари. Теория}
	
	\subsection{Основни вероятностни понятия}
	
	\begin{definition}  вероятност, събитие, благоприятно събитие: \\
		\textbf{Събитие, благоприятно събитие} \\ Ако вариантите за нещо да се случи са $n$, то всеки набор от тези $n$ изхода се нарича събитие. Благоприятно събитие може да наречем съвкупността от $"$желани$"$ изходи.\\
		\textbf{Вероятност} за едно събитие наричаме отношението благоприятни изходи/всички изходи на събитието\\

\end{definition}
	
	
\begin{definition} пермутации, комбинации, вариации : \\
	\textbf{Пермутация} \\
	\textbf{Вариация} - редът има значение\\
	\textbf{Комбинация} - редът няма значение  \\
	
\end{definition}


\begin{definition}
	случайна величина(непрекъсната и дискретна), разпределение, плътност \\
	 Дискретна случайна величина се задава с изходи $x_i$,които са с вероятности $p_i$, като $i$ може да е крайно, $i = 1,2 \dots n$, или безкарйно, но $"$изброимо$"$(може да броиш). Задава се с таблица(вж. по-долу)\\
	Непрекъснато разпределение имаме когато множеството от изходи $"$е интервал$"$, например $(0,1)$ 
\end{definition}

\begin{definition}
	Очакване, вариация ,стандартно отклонение\\
За случайна величина $X$ със стойности $x_i$ и вероятности $p_i$, дефинираме очакване като 
$$E(X) = \sum_i p_i x_i $$ 
\\
Вариация(Дисперсия) на $X$ дефинираме с:
$$ Var(X) = E(X^2) - (E(X))^2 $$
\end{definition}
	
	
	\section{Решавани задачи и техни подобни}
	
	\begin{problem}
	Пермутация на баби на пейка. 

	Пенка, Зинка, Любка \\
	ПЗЛ, ПЛЗ, ЗЛП, ЗПЛ, ЛПЗ, ПЗП \\


	\end{problem}	
	
	
	\begin{problem}
		Вариация телефонен номер
	\end{problem}
	
		\begin{problem}
		Комбинация избиране на топки от урна.\\ 
		$$ C_n^k = \frac{n!}{k!(n-k)!}  $$
		
		Нека имаме 2 сини и 3 червени топки.\\
		От 1 теглене, вероятността за синя е 2/5.\\
		От 2 тегления, да изтеглим синя и червена. \\
		
		Може първо да теглим синя, после червена, с  вероятност $\frac{2}{5} \cdot \frac{3}{4} = \frac{3}{10}$  \\
		или първо червена, после синя, с вероятност  $ \frac{3}{5} \cdot \frac{2}{4} = \frac{3}{10}$  \\
		Тогава при теглене на 2 топки, събираме вероятностите и получаваме
		$\frac{6}{10}$.\\
		Нека начините на избира на k топки от n топки означим(временно) с $C^k_n $. Тогава за начини на избиране на 1 синя топка имаме $C_2^1 $ и за $C_3^1 $. Да решим отново задачата с $C$-та и формулата благопрятни изходи върху всички изходи.
		Всички изходи за теглене на 2 топки от 5 са $C_5^2$. Умножаваме вариантите са теглене на 1 синя и 1 червена и делим на всички изходи, за да получим отговора. \\
		$$\frac{C^1_3 . C^1_2}{C_5^2} = \frac{3.2}{10} .$$
		$C_3^1 = \frac{3!}{2!1!} = 3 $, $C_2^1 = \frac{2!}{1!1!} = 2 $,  $C_5^2 = \frac{5!}{3!2!} = \frac{5.4}{2} = 10.$
		
		
		Този начин на решаване важи за всяка бройка червени и сини топки. Например, ако вместо по-горе, теглим 5 топки, искаме 2 сини, 3 червени от 30 топки(10 сини и 20 червени), то отговорът става:\\
		$$ \frac{C_{10}^2 . C_{20}^3}{C_{32}^5}. $$\\
		
	\end{problem}

		\begin{problem}
			Очакване на хвърляне на зар
			\begin{tabular}{|c|c c c c c c|} 
				\hline
				X & 1 & 2 & 3 & 4 & 5 & 6 \\ 
				\hline
				P & $\frac{1}{6}$ & $\frac{1}{6}$ & $\frac{1}{6}$ & $\frac{1}{6}$ & $\frac{1}{6}$ & $\frac{1}{6}$  \\ 
				\hline
			\end{tabular}
		\newline
		$\mathbb E(X) = \frac{1}{6}(1+2+3+4+5+6) = \frac{7}{2} = 3,5 $ \\
		Стандартно отклонение:
		\begin{tabular}{|c|c c c c c c|} 
			\hline
			$X^2$ & 1 & 4 & 9 & 16 & 25 & 36 \\ 
			\hline
			P & $\frac{1}{6}$ & $\frac{1}{6}$ & $\frac{1}{6}$ & $\frac{1}{6}$ & $\frac{1}{6}$ & $\frac{1}{6}$  \\ 
			\hline
		\end{tabular}
	\newline
		$\mathbb E(X^2) = \frac{1}{6}(1+4+9+16+25+36)  = \frac{1}{6}(91) = \frac{91}{6} $ \\
		$ Var(X) = E(X^2) - (E(X))^2 = \frac{91}{6} - (3,5)^2 = 15,166(6) - 12,25 \approx 2,92.$
	\end{problem}
	
	
	\begin{problem}
	Задача
		\begin{tabular}{|c|c c c c c c||} 
		\hline
		P&	0& 2&  50&  200&  1000&  \\
		\hline
		W&	x& 100&  20&  5&  1&  \\
		\hline
	\end{tabular}

\begin{problem}
	Задача за хвърляне на два зара
	\begin{tabular}{|c|c c c c c c|}
	  & 1 & 2 & 3 & 4 & 5 & 6 \\
		\hline
	 1  & 2 & 3 & 4 & 5 & 6 & 7 \\
	2 &	 3 & 4 & 5 & 6 & 7 & 8 \\
	3 &	4 & 5 & 6 & 7 & 8 & 9 \\
4 &		 5 & 6 & 7 & 8 & 9 & 10 \\
5 &	6 & 7 & 8 & 9 & 10 & 11 \\
6 &	7 & 8 & 9 & 10 & 11 & 12 \\

\end{tabular}
	
\end{problem}

(1,1), (1,2), (1,3)
	
	$ x = 20000- 1-5-20-100 = 19874$\\
	
	
	Очаквана печалба = (2.100 + 50.20 + 200.5 + 1.1000 ) / 20000 \\
	\end{problem}
	
	\begin{problem}[Задача за белот]
	
	Да се добави задача с карти - белот 
	$\frac{4}{32}\frac{3}{31}\frac{4}{30}\frac{3}{29}	$\\
	
	ДПДП	
	\end{problem}
	
	\vspace{4cm}
	
	
	\begin{problem} Гаусова елиминация(test)
		\begin{equation*}
			\systeme{
				3x +7z = 20,
				y - 17z = -3,
				24x + 15y = 7
			}
		\end{equation*}
	\end{problem}
	
	%%%%%%%%%%%%%%%%%%%%%%%%%
	\begin{elimination}[3]{3}{1.1em}{1.1}% Decreased from 1.75em
		\eliminationstep
		{
			4 & -8 & 5 & 1 & 0 & 0 \\
			4 & -7 & 4 & 0 & 1 & 0 \\
			3 & -4 & 2 & 0 & 0 & 1
		}
		{
			\\
			-R_{1} \\
			-\frac{3}{4} R_{1}
		}
		\eliminationstep
		{
			4 & -8 & 5 & 1 & 0 & 0 \\
			0 &  1 & -1 & -1 & 1 & 0 \\
			0 &  2 & -\frac{7}{4} & -\frac{3}{4} & 0 & 1
		}
		{
			\\
			\\
			-2R_{2}
		}
		\\[10pt]%                    increased spacing between rows
		\eliminationstep
		{
			4 & -8 & 5 & 1 & 0 & 0 \\
			0 &  1 & -1 & -1 & 1 & 0 \\
			0 &  0 & \frac{1}{4} & \frac{5}{4} & -2 & 1
		}
		{
			\\
			\\
			+2R_{2}
		}
	\end{elimination}
	
	Матрични уравнения: \\
	$AX = B $\\
	$X = A^{-1} B$
	
	
	%%%%%%%%%%%%%%%%%%%%%%%%%%
	
	
	\section{Въпроси и отговори}
	Задачата за линейно оптимиране само тази ли е? -Да. \\
	Какво трябва да правим по линейна алгебра? Само детерминанти и линейни системи? - Засега нищо.\\
	Детерминанти 2х2 или 3х3? - Може само да направим тренировъчен пример за пълнота.\\
	
	
	
	
	\end{document}