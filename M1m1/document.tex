\documentclass{article}
\usepackage[utf8]{inputenc}
\usepackage[bulgarian]{babel}

\usepackage{systeme}
\usepackage{amsmath}

\usepackage{cmap}
\usepackage[utf8]{inputenc}
\usepackage[T2A]{fontenc}

\newtheorem{definition}{Дефиниция}

\usepackage{comment}
%\use
\newtheorem{problem}{Задача}
\newtheorem{theorem}{Теорема}
\newcounter{solution}

\usepackage{graphicx}
\usepackage{pst-plot}

\usepackage{tikz}

\newcommand\solution{%
	\stepcounter{solution}%
	\textbf{Решение :}\\%
}

\usetikzlibrary{angles,
	quotes}
\usepackage{siunitx}

%%%%
\usepackage[margin=1in]{geometry}
\usepackage{mathtools}
\usepackage{array}

\makeatletter
\newcounter{elimination@steps}
\newcolumntype{R}[1]{>{\raggedleft\arraybackslash$}p{#1}<{$}}
\def\elimination@num@rights{}
\def\elimination@num@variables{}
\def\elimination@col@width{}
\newenvironment{elimination}[4][0]
{
	\setcounter{elimination@steps}{0}
	\def\elimination@num@rights{#1}
	\def\elimination@num@variables{#2}
	\def\elimination@col@width{#3}
	\renewcommand{\arraystretch}{#4}
	\start@align\@ne\st@rredtrue\m@ne
}
{
	\endalign
	\ignorespacesafterend
}
\newcommand{\eliminationstep}[2]
{
	\ifnum\value{elimination@steps}>0\sim\quad\fi
	\left[
	\ifnum\elimination@num@rights>0
	\begin{array}
		{@{}*{\elimination@num@variables}{R{\elimination@col@width}}
			|@{}*{\elimination@num@rights}{R{\elimination@col@width}}}
		\else
		\begin{array}
			{@{}*{\elimination@num@variables}{R{\elimination@col@width}}}
			\fi
			#1
		\end{array}
		\right]
		& 
		\begin{array}{l}
			#2
		\end{array}
		&%                                    moved second & here
		\addtocounter{elimination@steps}{1}
	}
	\makeatother


%%%%%



\date{}

\title{Хакване на някаква математика в УНСС}
\begin{document}
	\maketitle
	
	\section{Теми и коментари. Теория	}
	
	\subsection{Основни вероятностни понятия}
	вероятност, събитие, благоприятно събитие, пермутации, комбинации, вариации, \\
	случайна величина(непрекъсната и дискретна), разпределение, плътност, 
	
	
	
	\section{Решавани задачи и техни подобни}
	
	\begin{problem}
		\begin{equation*}
			\systeme{
				3x +7z = 20,
				y - 17z = -3,
				24x + 15y = 7
			}
		\end{equation*}
	\end{problem}
	
	%%%%%%%%%%%%%%%%%%%%%%%%%
	\begin{elimination}[3]{3}{1.1em}{1.1}% Decreased from 1.75em
		\eliminationstep
		{
			4 & -8 & 5 & 1 & 0 & 0 \\
			4 & -7 & 4 & 0 & 1 & 0 \\
			3 & -4 & 2 & 0 & 0 & 1
		}
		{
			\\
			-R_{1} \\
			-\frac{3}{4} R_{1}
		}
		\eliminationstep
		{
			4 & -8 & 5 & 1 & 0 & 0 \\
			0 &  1 & -1 & -1 & 1 & 0 \\
			0 &  2 & -\frac{7}{4} & -\frac{3}{4} & 0 & 1
		}
		{
			\\
			\\
			-2R_{2}
		}
		\\[10pt]%                    increased spacing between rows
		\eliminationstep
		{
			4 & -8 & 5 & 1 & 0 & 0 \\
			0 &  1 & -1 & -1 & 1 & 0 \\
			0 &  0 & \frac{1}{4} & \frac{5}{4} & -2 & 1
		}
		{
			\\
			\\
			+2R_{2}
		}
	\end{elimination}
	
	%%%%%%%%%%%%%%%%%%%%%%%%%%
	
	\begin{problem}
		content...
	\end{problem}
	
	
	1 2 3 4 5 6\\

	1/6\\
	
	(1-6 , 1-6)
	
	1 2 3 4 5 6\\
1	\\
2   \\
3 \\
4\\
5\\
6\\

Пенка, Зинка, Любка \\
ПЗЛ, ПЛЗ, ЗЛП, ЗПЛ, ЛПЗ, ПЗП \\

	
	
	$AX = B $\\
	
	$X = A^{-1} B$
	
	
	
	\begin{problem}
		
	\end{problem}
	
		\begin{problem}
		content...
	\end{problem}

		\begin{problem}
		content...
	\end{problem}
	
	\begin{table}[]
		\begin{tabular}{llllll}
			0& 2&  50&  200&  1000&  \\
			X& 100&  20&  5&  1&  \\
		\end{tabular}
	\end{table}
	$ X = 20000- 1-5-20-100 = 19874$\\
	
	Очаквана печалба = (2.100 + 50.20 + 200.5 + 1.1000 ) / 20000
	
	
	
	\section{Въпроси}
	Задачата за линейно оптимиране само тази ли е? \\
	Какво трябва да правим по линейна алгебра? Само детерминанти и линейни системи?\\
	Детерминанти 2х2 или 3х3?\\
	
	
	
	
	\end{document}